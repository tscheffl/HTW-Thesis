%%
%% Abstract
%% 120102 fschw: 1.Kurzfassung
%%
%%%%%%%%%%%%%%%%%%%%%%%%%%%%%%%%%%%%%%%%%%%%%%%%%%%%%%%%%%%%%%%%%%%%%


\section*{Kurzfassung}

Diese Arbeit beschreibt die Erstellung einer internetfähigen Steuerung für elektrische Verbraucher. Anforderungen an die Steuerung werden nach dem Kano-Modell definiert. Über eine Nutzwertanalyse werden vorhandene Techniken und Standards bewertet. Exemplarisch wird die Lösung mit dem größten Nutzwert implementiert.

Ein Zigbit-Modul, bestehend aus einem AVR Mikrocontroller und einem IEEE 802.15.4 Funkchip, bildet die Basis für die Hardware. Zusammen mit einem selbst dimensioniertem Kondensatornetzteil wird das Modul in einem Steckdosengehäuse verbaut.

Um zukunftssicher zu sein, wird das Protokoll IPv6 eingesetzt. Die Adaptionsschicht übernimmt das Protokoll 6LoWPAN. Das verwendete Betriebssystem Contiki besitzt eine fertige Webserver-Applikation, die für die eigenen Zwecke angepasst wird. Das Protokoll IEC 60870-5-104 wird neu implementiert. Es basiert auf dem TCP/IP-Modell und wird vor allem im Umfeld von Energieleitsystemen eingesetzt. Es eignet sich besonders für einen automatisierten Zugriff.

Über eine öffentliche Adresse des IPv6-Tunnelbrokers SixXS ist die Steuerung weltweit erreichbar und der elektrische Verbraucher kann über einen Webbrowser oder von einem Energieleitsystem ein- und ausgeschaltet werden.

Die Anforderungen nach dem Kano-Modell wurden nahezu vollständig erfüllt. Die Implementierung eines Webservers und einer IEC 60870-5-104 Applikation ist mit den gegebenen limitierten Ressourcen möglich. Anwendungsmöglichkeiten für die Steuerung liegen im Bereich eHome und Smart Grid.

%Motivation, Fragestellung, Methodik, Ergebnisse, Schlussfolgerungen

%Das heutige Internet kann man in zwei Teile ordnen. Das Core Internet, das aus Internet Routern und Servern besteht und darum herum das Fringe Internet, das z.B. aus Computern und Laptops besteht, mit denen sich Nutzer mit dem Internet verbinden. In Zukunft - und ansatzweise heute schon - wird es noch eine dritte Art geben: das Internet of Things, das aus sogenannten Smart Objects bestehen wird. Das sind kleine Mikrocontroller mit Sensoren und Aktoren aus vielen verschiedenen Bereichen.
%Diese Arbeit dreht sich um Smart Objects in der Heimautomatisierung. Es soll eine internetfähige Steuerung von elektrischen Verbrauchern realisiert werden. Grundlage dabei ist eine Mikrocontroller- Plattform mit einem IEEE 802.15.4 Funkübertragungsmodul. Auf diesem Mikrocontroller soll das 6LoWPAN-fähige Betriebssystem Contiki installiert werden. Von außen soll es möglich sein, sich mit dem Mikrocontroller über IPv6 zu verbinden und den elektrischen Verbraucher ein- oder auszuschalten.
%Die Arbeit teilt sich in mehrere Teile. Es muss eine geeignete Hardware ausgesucht oder entwickelt werden. Basis dafür ist das AVR RZ Raven Board von Atmel. Das Contiki Betriebssystem muss installiert werden. Die Mikrocontroller- Plattform muss über 6LoWPAN ansprechbar sein und eine geeignete Server Anwendung (z.b. Webserver oder SNMP Agent) muss so angepasst oder entwickelt werden, dass darüber der elektrische Verbraucher ein- oder ausgeschaltet werden kann.

%% eof
