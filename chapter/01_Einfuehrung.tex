%%
%% Beuth Hochschule für Technik --  Abschlussarbeit
%%
%% Kapitel 1
%%
%%

\chapter{Einleitung} \label{Einleitung}

\section{Aufgabenbeschreibung} \label{Aufgabenbeschreibung}

Im Zuge dieser Masterarbeit soll eine Lösung erarbeitet und entwickelt werden, mit der ein beliebiger elektrischer Verbraucher über einen Fernzugriff gesteuert werden kann. Eine im Handel erhältliche Steckdose soll so modifiziert werden, dass die Stromzufuhr des Verbrauchers über das Internet ein- und ausschaltbar ist. Für den Zugriff auf die Steckdose soll keine zusätzliche Verkabelung notwendig sein. Das vorhandene Steuergerät der Steckdose soll durch eine selbst entwickelte Mikrocontrollerschaltung ersetzt werden. Der Mikrocontroller soll eine Applikation bereitstellen, damit menschliche Benutzer möglichst einfach auf die Steckdose zugreifen können.

Es sollen verschiedene existierende Technologien, die für diese Anwendung in Frage kommen, recherchiert und bewertet werden.  Eine Lösung soll beispielhaft implementiert und nach einem Funktionstest kritisch begutachtet werden. Dadurch soll gezeigt werden, in wie weit es möglich ist sehr kleine Geräte mit besonders limitierten Ressourcen an ein lokales Netzwerk und an das Internet anzubinden.

\section{Herangehensweise}

In Kapitel \ref{Grundlagen} werden zuerst zentrale Begriffe, wie Smart Objects und Internet"=of"=Things, erläutert. Verschiedenen Technologien und Standards, die im Zusammenhang mit der Aufgabenstellung verwendet werden, werden vorgestellt. Diese Technologien werden eingeteilt in Übertragungstechniken und Kommunikationstechnologien. Eine Auswahl an Applikationen wird erläutert, die für den Zugriff auf die zu implementierende Lösung verwendet werden können. Betriebssysteme und Softwareumgebungen werden vorgestellt, die für die Bearbeitung der Aufgabenstellung in Frage kommen. Im Abschluss werden allgemein technische und nicht"=technische Herausforderungen behandelt.

Im Kapitel \ref{Konzept} wird anhand einer Nutzwertanalyse die Komplettlösung ermittelt, die für die Aufgabenstellung am geeignetsten ist.  Dazu werden zuerst, unabhängig von bestimmten Technologien, Anforderungen beschrieben, die an eine implementierte Lösung gestellt werden. Die Erfüllung dieser Anforderungen wird am Ende dieser Arbeit geprüft.  Am Ende wird die Entscheidung hergeleitet, welche Lösung beispielhaft implementiert wird.

Die Beschreibung der eigentlichen Implementierung findet sich in Kapitel \ref{Implementierung}.

Nach Abschluss der Implementierung wurden Funktionstests der implementierten Applikationen durchgeführt. Der Testaufbau und die Funktionstests werden in Kapitel \ref{Tests} behandelt.

Zum Ende der Arbeit wird die implementierte Lösung kritisch begutachtet. In Kapitel \ref{Fazit} wird geprüft, welche vorher beschriebenen Anforderungen erfüllt wurden. Der Nutzwert der implementierten Lösung wird ermittelt und mit dem Ergebnis der vorherigen Nutzwertanalyse verglichen. Verschiedene Anwendungsmöglichkeiten für die Implementierung und ähnliche Lösungsansätze werden behandelt. Ein Resümee der Arbeit wird gezogen.





